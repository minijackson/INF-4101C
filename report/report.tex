\documentclass{report}

\usepackage[utf8]{inputenc}
\usepackage[T1]{fontenc}
\usepackage[french]{babel}
\usepackage[hidelinks]{hyperref}
\usepackage{amsmath}
\usepackage{fullpage}
\usepackage{microtype}

\usepackage{fontspec}
\setmonofont[Scale=0.8]{Hack}
\usepackage{color}
\usepackage{minted}
\usepackage{booktabs}

\usemintedstyle{tomorrow}

\definecolor{bg}{rgb}{0.95,0.95,0.95}

\setminted{linenos,bgcolor=bg,tabsize=4,breaklines}
%\setmintedinline{bgcolor=white}

\renewcommand\listingscaption{Extrait de code}
\renewcommand\listoflistingscaption{Liste des extraits de code}

\author{Rémi \textsc{Nicole}}
\title{Rapport d'optimisation des temps de calcul et processeur}

\begin{document}

\maketitle

~\clearpage

\tableofcontents

% ====================
\chapter{Introduction}
% ====================

\section{Présentation du problème}

\paragraph{} Le but du projet demandé est de faire une partie d'un programme
servant à détecter des lignes, par exemple pour détecter le changement de voie
via une caméra embarquée dans une voiture.

\section{Architecture}

\paragraph{} Afin de pouvoir installer l'écosystème Rust, on m'a fourni une
carte plus récente, à savoir la carte Quad, Sabre Lite, de Freescale.

\paragraph{} L'architecture sur laquelle j'ai travaillé est une architecture
ARMfh, qui est donc une architecture supportant ARMv7 et ayant un support
matériel pour les points flottants (\textit{hf} signifiant
\textit{hard float}).

\paragraph{} De plus, la carte fournie possède un processeur contenant 2 cœurs
avec 1 thread par cœur, allant de 400MHz à 1000MHz.

\section{Choix du langage}

\paragraph{} Rust est un langage se focalisant sur la sécurité d'exécution: il
prévient les erreurs de segmentations, garantit la sûreté entre les threads,
tout en restant à une rapidité très proche du C++.

\paragraph{} La rapidité et la sécurité d'exécution correspondent tous deux aux
objectifs du projet, à savoir détecter rapidement les lignes et minimiser les
erreurs (utile si l'application est utilisé dans un contexte d'assistance à la
conduite).

\section{Bibliothèques en Rust utilisées}

\paragraph{} Voici l'ensemble des bibliothèques utilisées pour implémenter ce
projet en Rust suivies de leur description:

\begin{table}[H]
	\centering
	\begin{tabular}{lp{10cm}}
		\toprule
		Bibliothèque & Description\\
		\midrule
		piston\_window & Bibliothèque servant à afficher une fenêtre contenant
		un contexte graphique 2D ou 3D.\\

		image & Bibliothèque fournissant un écosystème pour le traitement
		d'image avec des fonctions pour lire, écrire et convertir des images.
		Quelques fonctions de traitement d'image sont présentes (flou gaussien,
		recadrages\ldots) mais ne sont pas utilisées dans le projet.\\

		rscam & Bibliothèque se basant sur la bibliothèque C v4l pour fournir
		un moyen de récupérer des images venant de caméras.\\

		docopt & Bibliothèque permettant de lire d'une manière simple les
		arguments fournis via la ligne de commande.\\
		\bottomrule
	\end{tabular}%
	\label{tab:libs}
	\caption{Bibliothèques utilisées pour ce projet}
\end{table}

% ======================
\chapter{Projet de base}
% ======================

\paragraph{} Afin de bien pouvoir commencer le projet, il m'a fallu analyser le
code du projet initial qui nous a été donné et qui utilise la librairie OpenCV
pour fournir une implémentation basique. Il m'a fallu ensuite traduire ça en
Rust pour pouvoir l'optimiser.

\section{Analyse du code}

\paragraph{} On remarque que le projet de base essaye tout d'abord d'ouvrir le
flux vidéo de la caméra ayant l'identifiant 0, puis affiche 4 fenêtres, et
calcule le résultat du filtrage de l'image actuelle par les filtres conversion
en niveau de gris, sobel et médian. Ensuite, il les affiches côte à côte avec
l'image originale, et ce,tant que l'utilisateur n'appuie pas sur la touche
``q''.

\paragraph{} On peut voir que toutes les fonctions de filtres, affichage, et
récupération de l'image de la webcam et même attente des entrées du clavier
sont fournies par OpenCV.

\section{Défauts d'optimisations}

\paragraph{} Sans regarder l'implémentation d'OpenCV des filtres, on peut au
premier abord remarquer que le filtre Sobel est effectué en cinq étapes
séparées:

\begin{itemize}
	\item Calcul du gradient X
	\item Calcul de la valeur absolue du gradient X
	\item Calcul du gradient Y
	\item Calcul de la valeur absolue du gradient Y
	\item Calcul du gradient total
\end{itemize}

\paragraph{} Comme il s'agit de cinq appels séparés, on peut en déduire que
pour le filtrage par le filtre Sobel d'une seule image, il nous faut itérer
cinq fois sur chaque pixel de l'image. Comme il est possible d'avoir le même
résultat en un seul parcours de l'image, il s'agit donc bien d'une optimisation
possible.

\paragraph{} De plus, d'une manière plus générale, les filtres sont tous
effectués séparément, alors qu'il est possible de les combiner pour minimiser
le nombre de parcours des pixels de l'image.

\paragraph{} Enfin, les implémentations d'OpenCV sont faites de telle sorte
qu'elles s'exécutent sur un seul thread, et donc ne tirent pas d'avantage des
machines possédant plusieurs cœurs ou ayant la fonctionnalité
d'hyper-threading.

\section{Défauts de style}

\paragraph{} Bien que le C++ soit presque complètement rétro-compatible avec le
C, l'utilisation des header C fournis dans la glibc sont déconseillés. En
effet, plutôt que d'utiliser ``stdio.h'' et ``stdlib.h'', il est préférable
d'utiliser ``cstdio'' et ``cstdlib'' qui sont fournis par le C++. De plus,
aucune des fonctions venant de ces headers sont utilisées. Ils peuvent donc
être supprimés.

\paragraph{} Enfin, selon l'utilisation des directives
\mintinline{cpp}{using namespace} sont déconseillées (notamment par les
conventions de code de Google) pour éviter de ``polluer'' l'espace de nom
global. Il est préférable d'utiliser des directives telles que
\mintinline{cpp}{using std::cout;} ou \mintinline{cpp}{using cv::Mat;}
uniquement.

% ================================================
\chapter{Implémentation du projet de base en Rust}
% ================================================

\paragraph{} L'étape suivante après l'analyse du projet de base fournit a été
de trouver et implémenter un projet équivalent en Rust.

\section{Boucle principale}

\paragraph{} La bibliothèque \texttt{piston\_window} fournit une boucle
principale dans laquelle le traitement s'opère. La méthode utilisée ressemble à
celle-ci:\\

\begin{listing}[H]
\begin{minted}{rust}
let (sender, receiver) = channel();

capture::stream(sender, device);

while let Some(e) = window.next() {
	if let Ok(frame) = receiver.try_recv() {
		let frame = process_frame(frame.convert(), median_size, threshold);
		displayed_texture = display::build_texture(/* ...frame... */);
	}

	display::show_texture(/* ...displayed_texture... */);
}
\end{minted}
\label{lst:main_loop}
\caption{Boucle principale}
\end{listing}

\paragraph{} Afin de traiter en temps réel le flux venant de la caméra, nous
démarrons un thread capturant les images de la webcam et les envoyant au thread
principal via un \texttt{channel} et on traite l'image dans le thread
principal.

\section{Capture des images de la webcam}

\paragraph{} Voici une partie du code simplifiée permettant de récupérer une
image de la caméra.

\begin{listing}[H]
\begin{minted}{rust}
let mut cam = Camera::new(device).unwrap();

cam.start(&Config {
	interval: (1, 30),
	resolution: (640, 480),
	format: b"RGB3",
	..Default::default()
}).unwrap();

let frame = cam.capture().unwrap();
let frame : ImageBuffer<Rgb<u8>, _>
	= ImageBuffer::from_raw(frame.resolution.0,
	                        frame.resolution.1,
	                        frame).unwrap();
return frame.convert();
\end{minted}
\label{lst:capture}
\caption{Capture d'une image}
\end{listing}

\paragraph{} La première partie, lignes 1--8, sert à initialiser la caméra avec
certains paramètres. \mintinline{rust}{device} est une variable contenant sous
la forme d'une chaîne de caractères le ``device'' représentant la caméra dans
les systèmes POSIX. Elle sera la plupart du temps égale à
\mintinline{rust}{"/dev/video0"}.

\paragraph{} Ensuite, nous démarrons la caméra avec certains paramètres:

\begin{description}
	\item[interval:] Une fréquence d'images fournie sous la forme d'intervalle,
		à savoir ici $\frac{1}{30}$ secondes entre chaque images.
	\item[resolution:] Une taille définie en pixels
	\item[format:] Le format pour le stockage des couleurs, fournie sous la
		convention ``FourCC''.
	\item Les paramètres restants sont initialisés à leurs valeurs par défaut.
		Parmi ces paramètres, il y a la méthode de stockage pour les vidéos
		entrelacées et le nombre de buffers utilisés par la caméras.
\end{description}

\paragraph{} Puis, nous capturons l'image, et convertissons le résultat dans un
format convenable pour le traiter grâce à l'écosystème fourni par la
bibliothèque ``image'' grâce à la fonction
\mintinline{rust}{ImageBuffer::from_raw}. Cette fonction nous retourne donc
une structure de type \mintinline{rust}{ImageBuffer<Rgb<u8>, _>},
\mintinline{rust}{_} étant un type automatiquement déduit, et sera ensuite
convertie en \mintinline{rust}{ImageBuffer<Luma<u8>, Vec<u8>>} au retour de
la fonction. Cela a pour effet de convertir l'image couleur en niveaux de
luminosités (\mintinline{rust}{Luma<u8>}) et de stocker les pixels sous la
forme d'un tableau d'entiers non-signés de 1 octet
(\mintinline{rust}{Vec<u8>}).

\paragraph{} Nous pouvons remarquer dans ce codes des méthodes
\mintinline{rust}{.unwrap()} disséminés un peu partout. C'est tout simplement
dû au fait que presque chaque opération peut ne pas s'effectuer correctement.
Par exemple à la ligne 1, le fichier ``device'' peut ne pas exister, aux lignes
3--8, la caméra peut ne pas supporter certaines valeurs de paramètres, etc.
Cette méthode permet de dire qui si telle action s'est mal exécutée, le
programme doit immédiatement s'arrêter.

\section{Traitement de l'image}

\subsection{Fonction principale}

\paragraph{} Pour pouvoir réutiliser le traitement global de l'image dans
d'autres fonctions, comme pour le benchmarking ou pour d'autre moyens
d'acquisitions / visualisations, le traitement d'image est stocké dans une
fonction telle que voici:\\

\begin{listing}[H]
\begin{minted}{rust}
fn process_frame(frame : ImageBuffer<Luma<u8>, Vec<u8>>,
                 median_kernel_size : usize,
                 threshold : u8)
		-> ImageBuffer<Luma<u8>, Vec<u8>> {
	let frame = processing::median_filter(frame, median_kernel_size);
	let frame = processing::sobel(frame);
	return processing::threshold(frame, threshold);
}
\end{minted}
\label{lst:main_process}
\caption{Traitement d'une image}
\end{listing}

\paragraph{} On remarque que les fonctions effectuant les filtres sont préfixés
par \mintinline{rust}{processing::}. Il s'agit d'un module, équivalent aux
``namespaces'' en C++, dans lequel j'ai mis les fonctions de traitement
d'image, pour des raisons de clarté.

\paragraph{} On notera aussi que la variable \mintinline{rust}{frame} a été
redéclarée plusieurs fois, plutôt que réassigné, car lorsqu'on l'utilise par
exemple dans la fonction \mintinline{rust}{processing::median_filter}, la
variable frame n'est pas copiée, mais ``déplacée'' dans la fonction et ainsi
n'est plus utilisable après cette étape. Il nous faut donc stocker le résultat
dans une nouvelle variable. Heureusement, le langage Rust nous permet de
réutiliser des noms de variables que ce soient des types équivalents auparavant
ou non.

\subsection{Implémentation naïve du filtre Sobel}

\paragraph{} Voici l'implémentation naïve du filtre Sobel:\\

\begin{listing}[H]
\begin{minted}{rust}
pub fn sobel(frame : ImageBuffer<Luma<u8>, Vec<u8>>)
		-> ImageBuffer<Luma<u8>, Vec<u8>> {
	let mut result = ImageBuffer::new(640, 480);

	for i in 1..639 {
		for j in 1..479 {
			let north_west = frame[(i-1, j-1)].channels()[0] as i32;
			let north      = frame[(i, j-1)].channels()[0] as i32;
			let north_east = frame[(i+1, j-1)].channels()[0] as i32;

			let west = frame[(i-1, j)].channels()[0] as i32;
			let east = frame[(i+1, j)].channels()[0] as i32;

			let south_west = frame[(i-1, j+1)].channels()[0] as i32;
			let south      = frame[(i, j+1)].channels()[0] as i32;
			let south_east = frame[(i+1, j+1)].channels()[0] as i32;

			let gx : i32 = north_west + south_west + (2 * west)  -
			               north_east - south_east - (2 * east);
			let gy : i32 = north_west + north_east + (2 * north) -
			               south_west - south_east - (2 * south);

			let root : u8 = (((gx * gx) + (gy * gy)) as f32).sqrt() as u8;

			result.put_pixel(i, j, Luma([root]));
		}
	}

	return result;
}
\end{minted}
\label{lst:naive_sobel}
\caption{Sobel naïf}
\end{listing}

\paragraph{} On remarque qu'il y a beaucoup de conversions explicites entre
entiers signés / non-signés, flottants, et tailles 1 octet / 4 octets. Cela
s'explique par le fait qu'en Rust, il n'y ait pas de conversion implicites pour
les nombres, et que par exemple, aux lignes 18--21, si les valeurs n'étaient
pas des entiers signés, les nombres négatifs ne seraient pas atteignables: la
soustraction effectuerait un ``underflow'' et ferait planter le programme
(contrairement au C / C++ où le programme continuerait dans son erreur).

\paragraph{} On observe aussi la conversion en \mintinline{rust}{f32} juste
avant la racine carrée pour garder une précision convenable, suivie d'une
conversion en \mintinline{rust}{u8} pour représenter une valeur d'un niveau
de luminosité.

\paragraph{} On peut voir aussi des appels à des méthodes
\mintinline{rust}{.channels()} puis une récupération de l'élément à l'indice 0.
Ces opérations sont présentes car une image peut être composée de couleurs,
stockées sur 3 canaux, avoir une valeur d'alpha, rajoutant un canal, etc.
La luminosité n'étant stockée que sur un canal, seul l'élément à l'indice 0
existe. Cela explique aussi pourquoi la création d'un luminosité:
\mintinline{rust}{Luma([root])} soit composée d'un tableau dans les paramètres.

\paragraph{} Hormis ces particularités, il s'agit d'un filtre Sobel directement
traduit du début du TP2.

\subsection{Implémentation naïve du filtre médian}

\paragraph{} Voici l'implémentation naïve du filtre médian $3 \times 3$:

\begin{listing}[H]
\begin{minted}{rust}
pub fn median_filter(frame : ImageBuffer<Luma<u8>, Vec<u8>>)
		-> ImageBuffer<Luma<u8>, Vec<u8>> {
	let mut result = ImageBuffer::new(640, 480);

	let mut kernel = [0; 9];

	for i in 1..638 {
		for j in 1..478 {
			// Fill the kernel
			for k in 0..3 {
				for l in 0..3 {
					let index = k + 3 * l;
					let coord_x = (i + k - 1) as u32;
					let coord_y = (j + l - 1) as u32;

					kernel[index] = frame[(coord_x, coord_y)].channels()[0];
				}
			}

			kernel.sort();
			let pixel_value = kernel[5];
			result.put_pixel(i as u32, j as u32, Luma([pixel_value]));
		}
	}

	return result;
}
\end{minted}
\label{lst:naive_median}
\caption{Filtre médian naïf}
\end{listing}

\paragraph{} Il s'agit ici de l'implémentation ayant l'algorithme utilisant le
tri pour calculer les valeurs médianes. Nous avons donc une complexité totale
de:

\[
	O\left(\overbrace{n}^\text{Parcours de l'image} \times
	  \left(\underbrace{n \times \log(n)}_\text{Tri du tableau} +
	  \overbrace{n}^\text{Ajout des couleurs}\right)\right) = O(n^2 \times \log(n))
\]

\paragraph{} Rien de particulièrement extraordinaire ici: pour chaque pixel,
on remplit un tableau avec les contours du pixel, on le trie puis on récupère
la valeur médiane et on le stocke dans l'image de sortie aux mêmes coordonnées.

% =====================
\chapter{Optimisations}
% =====================

\section{Sobel}

\subsection{Optimisations effectuées}

\paragraph{} L'optimisation du sobel est la même que dans le TP2: on déroule la
boucle une fois, puis on fait une approximation du calcul du gradient. Voici
le résultat:

\begin{listing}[H]
\begin{minted}{rust}
pub fn sobel_optimized(frame : ImageBuffer<Luma<u8>, Vec<u8>>)
		-> ImageBuffer<Luma<u8>, Vec<u8>> {
	let mut result = ImageBuffer::new(640, 480);

	let mut i = 1;
	while i < 638 {
		let mut j = 1;
		while j < 479 {
			let north_west  = frame[(i-1, j-1)].channels()[0] as i32;
			let north       = frame[(i, j-1)].channels()[0] as i32;
			let north_east  = frame[(i+1, j-1)].channels()[0] as i32;
			let north_east2 = frame[(i+2, j-1)].channels()[0] as i32;

			let west  = frame[(i-1, j)].channels()[0] as i32;
			let west2 = frame[(i, j)].channels()[0] as i32;
			let east  = frame[(i+1, j)].channels()[0] as i32;
			let east2 = frame[(i+2, j)].channels()[0] as i32;

			let south_west  = frame[(i-1, j+1)].channels()[0] as i32;
			let south       = frame[(i, j+1)].channels()[0] as i32;
			let south_east  = frame[(i+1, j+1)].channels()[0] as i32;
			let south_east2 = frame[(i+2, j+1)].channels()[0] as i32;

			let gx : i32 = north_west + south_west + (west << 1)  -
			               north_east - south_east - (east << 1);
			let gy : i32 = north_west + north_east + (north << 1) -
			               south_west - south_east - (south << 1);

			let gx2 : i32 = north + (west2 << 1) + south -
			                north_east2 - (east2 << 1) - south_east2;
			let gy2 : i32 = north + (north_east << 1) + north_east2 -
			                south - (south_east << 1) - south_east2;

			let root  : u8 = (((gx.abs()  + gy.abs())  >> 1) as f32 * 1.414216) as u8;
			let root2 : u8 = (((gx2.abs() + gy2.abs()) >> 1) as f32 * 1.414216) as u8;

			result.put_pixel(i, j, Luma([root]));
			result.put_pixel(i + 1, j, Luma([root2]));

			j += 1;
		}
		i += 2;
	}

	return result;
}
\end{minted}
\label{lst:sobel_optimized}
\caption{Sobel optimisé}
\end{listing}

\paragraph{} La seule particularité induite par le langage Rust a été de devoir
changer les boucles \mintinline{rust}{for} en boucles \mintinline{rust}{while}
pour pouvoir incrémenter par 2. En effet, les boucles \mintinline{rust}{for} en
Rust sont l'équivalent des boucles ``foreach'' en C++ ou Java.

\subsection{Gains}

% TODO(minijackson)
\paragraph{}

\begin{figure}[H]
	\centering
	\includegraphics[width=\textwidth]{sobels.pdf}%
	\label{fig:sobels}
	\caption{Comparaisons des différentes implémentations du filtre Sobel}
\end{figure}

\section{Médian}

\subsection{Avec histogramme simple}

\subsubsection{Optimisations effectuées}

\paragraph{} La première étape pour optimiser le filtre médian a été d'utiliser
un histogramme pour calculer les médians plutôt que de devoir faire un tri
entier.

\paragraph{} La méthode simple est de créer un histogramme pour le pixel
actuel, puis de le remplir avec les valeurs de luminosité des pixels au
voisinage. Nous avons donc un histogramme de taille 255, et la méthode pour
calculer le médian est de prendre le nombre d'éléments $s$ de l'histogramme
(taille du filtre au carré dans les cas hors du bord), et de le soustraire
successivement au nombre d'élément dans une colonne. En faisant varier les
indices de colonne dans l'ordre croissant, lorsque $s = 0$, c'est l'indice
actuel qui est le médian de l'histogramme. En code Rust, cela donne:

\begin{listing}[H]
\begin{minted}{rust}
pub fn median(&self) -> u8 {
	let mut sum : i32 = self.count as i32 / 2;
	let mut index = 0;

	while sum > 0 && index < 256 {
	    sum -= self.values[index] as i32;
	    index += 1;
	}

	return (index - 1) as u8;
}
\end{minted}
\label{tab:hist_median}
\caption{Calcul du médian d'un histogramme}
\end{listing}

\paragraph{} On remarque que le premier paramètre n'est pas typé et s'appelle
\mintinline{rust}{self}, qui est un mot-clef particulier en Rust. Il s'agit
d'une définition d'une ``méthode'' de la structure \mintinline{rust}{Histogram}
et self représente l'objet actuel (équivalent à \mintinline{cpp}{this} en C++).

\paragraph{} Afin de spécifier qu'il s'agissent de méthodes de la structure
\mintinline{rust}{Histogram}, les définitions sont entourées d'un bloc
\mintinline{rust}{impl Histogram}. Parmi les autres méthode, il y a
\mintinline{rust}{increment(luma)}, \mintinline{rust}{decrement(luma)} pour
incrémenter et décrémenter des valeurs de luminosités, et une méthode
``statique'' \mintinline{rust}{new()} pour créer un nouvel histogramme.

\paragraph{} Au final, cette méthode nous donne comme complexité:

\[
	O\left(\overbrace{n}^\text{Parcours de l'image} \times
	  \left(\underbrace{n}_\text{Ajout des couleurs} +
	  \overbrace{n}^\text{Calcul du médian}\right)\right) = O(n^2)
\]

\paragraph{} On a donc pour fonction de filtrage par filtre médian:

\begin{listing}[H]
\begin{minted}{rust}
pub fn median_filter_hist(frame : ImageBuffer<Luma<u8>, Vec<u8>>,
                          kernel_size : usize)
		-> ImageBuffer<Luma<u8>, Vec<u8>> {
	let mut result = ImageBuffer::new(640, 480);

	let kernel_offset = ((kernel_size - 1) / 2) as i32;

	for i in 0..640 {
		for j in 0..480 {
			let mut hist = Histogram::new();

			for k in (i as i32 - kernel_offset)..(i as i32 + kernel_offset + 1) {
				for l in (j as i32 - kernel_offset)..(j as i32 + kernel_offset + 1) {
					if check_coordinates(k, l) {
						let color = frame[(k as u32, l as u32)].channels()[0];
						hist.increment(color);
					}
				}
			}

			let median_color = Luma([hist.median()]);
			result.put_pixel(i as u32, j as u32, median_color);
		}
	}

	return result;
}
\end{minted}
\label{lst:median_filter_hist}
\caption{Filtre médian optimisé avec histogramme en $O(n^2)$}
\end{listing}

\subsubsection{Gains}

\begin{figure}[H]
	\centering
	\includegraphics[width=\textwidth]{simple_medians.pdf}%
	\label{fig:simple_medians}
	\caption{Comparaisons des deux implémentations du filtre médian}
\end{figure}

\subsection{Avec histogramme optimisé}

\subsubsection{Optimisations effectuées}

\paragraph{} L'étape suivante pour l'optimisation du filtre médian a été
d'implémenter l'algorithme décrit dans ``\textit{Median Filtering in Constant
	Time}''. Il s'agit donc de réutiliser le précédent histogramme pour éviter
de insérer à nouveau les valeurs de luminosités déjà récupérées.

\paragraph{} Nous avons donc un algorithme de complexité:

\[
	O\left(\overbrace{n}^\text{Parcours de l'image} \times
	  \underbrace{2 \times \sqrt{n}}_\text{\parbox{3cm}{Ajout et suppression des colonnes de l'histogramme}} +
	  \overbrace{\sqrt{n}}^\text{\parbox{3cm}{Nombre d'initialisations d'histogrammes}} \times
	  \underbrace{n}_\text{Taille du filtre}\right) = O(n\sqrt{n})
\]

\paragraph{} Et voici comment cela se traduit dans le projet:

\begin{listing}[H]
\begin{minted}{rust}
pub fn median_filter_hist_optimized(frame : ImageBuffer<Luma<u8>, Vec<u8>>,
                                    kernel_size : usize)
		-> ImageBuffer<Luma<u8>, Vec<u8>> {
	let mut result = ImageBuffer::new(640, 480);

	let kernel_offset = ((kernel_size - 1) / 2) as i32;

	for i in 0..640 {
		let mut hist = Histogram::new();

		for k in (i as i32 - kernel_offset)..(i as i32 + kernel_offset + 1) {
			for l in 0..(kernel_offset + 1) {
				if check_coordinates(k, l as i32) {
					let color = frame[(k as u32, l as u32)].channels()[0];
					hist.increment(color);
				}
			}
		}

		for j in 0..480 {
			let old_column_coord = j as i32 - kernel_offset - 1i32;
			let new_column_coord = j as i32 + kernel_offset;

			for k in (i as i32 - kernel_offset)..(i as i32 + kernel_offset + 1) {
				if check_coordinates(k, old_column_coord) {
					let color = frame[(k as u32, old_column_coord as u32)].channels()[0];
					hist.decrement(color);
				}
			}

			for k in (i as i32 - kernel_offset)..(i as i32 + kernel_offset + 1) {
				if check_coordinates(k, new_column_coord) {
					let color = frame[(k as u32, new_column_coord as u32)].channels()[0];
					hist.increment(color);
				}
			}

			let median_color = Luma([hist.median()]);
			result.put_pixel(i as u32, j as u32, median_color);
		}
	}

	return result;
}
\end{minted}
\label{lst:median_filter_hist_opti}
\caption{Filtre médian optimisé avec histogramme en $O(n\sqrt{n})$}
\end{listing}

\subsubsection{Gains}

\begin{figure}[H]
	\centering
	\includegraphics[width=\textwidth]{medians.pdf}%
	\label{fig:medians}
	\caption{Comparaisons des trois implémentations du filtre médian}
\end{figure}

% TODO(minijackson)
\paragraph{}

\section{Combinaisons}

\subsection{Optimisations effectuées}

\paragraph{}

\subsection{Gains}

% TODO(minijackson)
\paragraph{}

% ==================
\chapter{Benchmarks}
% ==================

% TODO(minijackson)
\paragraph{} Commodité

\begin{table}[H]
	\centering
	\begin{tabular}{lrr}
		\toprule
		Benchmark & Nanosecondes par itération (ns) & Écart type (ns)\\
		\midrule
		Filtre médian naïf taille 3                & 156,249,733   & +/- 2,564,199\\
		\midrule
		Filtre médian Histogramme $O(n)$ taille 3  & 273,835,633   & +/- 2,628,886\\
		Filtre médian Histogramme $O(n)$ taille 5  & 350,479,233   & +/- 2,987,593\\
		Filtre médian Histogramme $O(n)$ taille 9  & 632,823,133   & +/- 2,601,370\\
		Filtre médian Histogramme $O(n)$ taille 15 & 1,316,361,433 & +/- 6,738,010\\
		Filtre médian Histogramme $O(n)$ taille 21 & 2,308,687,366 & +/- 7,455,609\\
		Filtre médian Histogramme $O(n)$ taille 31 & 4,294,967,295 & +/- 31,194,853\\
		\midrule
		Filtre médian Histogramme $O(1)$ taille 3  & 213,110,199   & +/- 2,564,106\\
		Filtre médian Histogramme $O(1)$ taille 5  & 231,504,033   & +/- 1,810,410\\
		Filtre médian Histogramme $O(1)$ taille 9  & 267,983,399   & +/- 3,825,399\\
		Filtre médian Histogramme $O(1)$ taille 15 & 321,735,000   & +/- 3,592,493\\
		Filtre médian Histogramme $O(1)$ taille 21 & 381,676,533   & +/- 4,287,863\\
		Filtre médian Histogramme $O(1)$ taille 31 & 470,031,866   & +/- 3,173,656\\
		\midrule
		Sobel naïf                                 &  29,704,633   & +/- 1,493,210\\
		Sobel optimisé                             &  19,239,366   & +/- 1,094,539\\
		Sobel optimisé et seuillage                &  21,553,533   & +/- 1,366,083\\
		\midrule
		Traitement complet avec médian taille 3    & 236,703,566   & +/- 3,445,960\\
		Traitement complet avec médian taille 5    & 254,697,900   & +/- 3,184,183\\
		Traitement complet avec médian taille 9    & 290,787,933   & +/- 3,738,697\\
		Traitement complet avec médian taille 15   & 344,906,099   & +/- 4,363,739\\
		Traitement complet avec médian taille 21   & 405,945,233   & +/- 4,189,213\\
		Traitement complet avec médian taille 31   & 493,546,833   & +/- 5,892,350\\
		\bottomrule
	\end{tabular}%
	\label{tab:bench}
	\caption{Résultats des benchmarks sur la carte}
\end{table}

\begin{figure}[H]
	\centering
	\includegraphics[width=\textwidth]{process.pdf}%
	\label{fig:process}
	\caption{Benchmarks des implémentations optimisées}
\end{figure}

% ==================
\chapter{Conclusion}
% ==================

\paragraph{} TODO: Dans une version ultérieure.

% ====================
\chapter{Perspectives}
% ====================

\paragraph{} TODO: Dans une version ultérieure.

\listoffigures
\listoflistings
\listoftables

\end{document}
